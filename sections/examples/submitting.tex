Submission to ICML 2015 will be entirely electronic, via a web site
(not email).  Information about the submission process and \LaTeX\ templates
are available on the conference web site at:
\begin{center}
\textbf{\texttt{http://icml.cc/2015/}}
\end{center}
Send questions about submission and electronic templates to
\texttt{program@icml.cc}.

The guidelines below will be enforced for initial submissions and
camera-ready copies.  Here is a brief summary:
\begin{itemize}
\item Submissions must be in PDF.
\item The maximum paper length is \textbf{8 pages excluding references, and 10 pages
  including references} (pages 9 and 10 must contain only references).
\item Do \textbf{not include author information or acknowledgments} in your initial
submission. 
\item Your paper should be in \textbf{10 point Times font}.
\item Make sure your PDF file only uses Type-1 fonts.
\item Place figure captions {\em under} the figure (and omit titles from inside
the graphic file itself).  Place table captions {\em over} the table.
\item References must include page numbers whenever possible and be as complete
as possible.  Place multiple citations in chronological order.  
\item Do not alter the style template; in particular, do not compress the paper
format by reducing the vertical spaces.
\end{itemize}

\subsection{Submitting Papers}

{\bf Paper Deadline:} The deadline for paper submission to ICML 2015
is at \textbf{23:59 Universal Time (3:59 Pacific Daylight Time) on February 6, 2015}.
If your full submission does not reach us by this time, it will 
not be considered for publication. There is no separate abstract submission.

{\bf Anonymous Submission:} To facilitate blind review, no identifying
author information should appear on the title page or in the paper
itself.  Section~\ref{author info} will explain the details of how to
format this.

{\bf Simultaneous Submission:} ICML will not accept any paper which,
at the time of submission, is under review for another conference or
has already been published. This policy also applies to papers that
overlap substantially in technical content with conference papers
under review or previously published. ICML submissions must not be
submitted to other conferences during ICML's review period. Authors
may submit to ICML substantially different versions of journal papers
that are currently under review by the journal, but not yet accepted
at the time of submission. Informal publications, such as technical
reports or papers in workshop proceedings which do not appear in
print, do not fall under these restrictions.

\medskip

To ensure our ability to print submissions, authors must provide their
manuscripts in \textbf{PDF} format.  Furthermore, please make sure
that files contain only Type-1 fonts (e.g.,~using the program {\tt
  pdffonts} in linux or using File/DocumentProperties/Fonts in
Acrobat).  Other fonts (like Type-3) might come from graphics files
imported into the document.

Authors using \textbf{Word} must convert their document to PDF.  Most
of the latest versions of Word have the facility to do this
automatically.  Submissions will not be accepted in Word format or any
format other than PDF. Really. We're not joking. Don't send Word.

Those who use \textbf{\LaTeX} to format their accepted papers need to pay close
attention to the typefaces used.  Specifically, when producing the PDF by first
converting the dvi output of \LaTeX\ to Postscript the default behavior is to
use non-scalable Type-3 PostScript bitmap fonts to represent the standard
\LaTeX\ fonts. The resulting document is difficult to read in electronic form;
the type appears fuzzy. To avoid this problem, dvips must be instructed to use
an alternative font map.  This can be achieved with the following two commands:

{\footnotesize
\begin{verbatim}
dvips -Ppdf -tletter -G0 -o paper.ps paper.dvi
ps2pdf paper.ps
\end{verbatim}}
Note that it is a zero following the ``-G''.  This tells dvips to use
the config.pdf file (and this file refers to a better font mapping).

A better alternative is to use the \textbf{pdflatex} program instead of
straight \LaTeX. This program avoids the Type-3 font problem, however you must
ensure that all of the fonts are embedded (use {\tt pdffonts}). If they are
not, you need to configure pdflatex to use a font map file that specifies that
the fonts be embedded. Also you should ensure that images are not downsampled
or otherwise compressed in a lossy way.

Note that the 2015 style files use the {\tt hyperref} package to
make clickable links in documents.  If this causes problems for you,
add {\tt nohyperref} as one of the options to the {\tt icml2015}
usepackage statement.

\subsection{Reacting to Reviews}

We will continue the ICML tradition in which the authors are given the
option of providing a short reaction to the initial reviews. These
reactions will be taken into account in the discussion among the
reviewers and area chairs.

\subsection{Submitting Final Camera-Ready Copy}

The final versions of papers accepted for publication should follow the
same format and naming convention as initial submissions, except of
course that the normal author information (names and affiliations)
should be given.  See Section~\ref{final author} for details of how to
format this.

The footnote, ``Preliminary work.  Under review by the International
Conference on Machine Learning (ICML).  Do not distribute.'' must be
modified to ``\textit{Proceedings of the
$\mathit{31}^{st}$ International Conference on Machine Learning},
Lille, France, 2015.  JMLR: W\&CP volume 37. 
Copyright 2015 by the author(s).''

For those using the \textbf{\LaTeX} style file, simply change
$\mathtt{\backslash usepackage\{icml2015\}}$ to 

$$\mathtt{\backslash usepackage[accepted]\{icml2015\}}$$

\noindent
Authors using \textbf{Word} must edit the
footnote on the first page of the document themselves.

Camera-ready copies should have the title of the paper as running head
on each page except the first one.  The running title consists of a
single line centered above a horizontal rule which is $1$ point thick.
The running head should be centered, bold and in $9$ point type.  The
rule should be $10$ points above the main text.  For those using the
\textbf{\LaTeX} style file, the original title is automatically set as running
head using the {\tt fancyhdr} package which is included in the ICML
2015 style file package.  In case that the original title exceeds the
size restrictions, a shorter form can be supplied by using

\verb|\icmltitlerunning{...}|

just before $\mathtt{\backslash begin\{document\}}$.
Authors using \textbf{Word} must edit the header of the document themselves.
